%\input{tcilatex}
%\input{tcilatex}
%\input{tcilatex}
%\input{tcilatex}


\documentclass[12pt]{article}
%%%%%%%%%%%%%%%%%%%%%%%%%%%%%%%%%%%%%%%%%%%%%%%%%%%%%%%%%%%%%%%%%%%%%%%%%%%%%%%%%%%%%%%%%%%%%%%%%%%%%%%%%%%%%%%%%%%%%%%%%%%%%%%%%%%%%%%%%%%%%%%%%%%%%%%%%%%%%%%%%%%%%%%%%%%%%%%%%%%%%%%%%%%%%%%%%%%%%%%%%%%%%%%%%%%%%%%%%%%%%%%%%%%%%%%%%%%%%%%%%%%%%%%%%%%%
\usepackage[T1]{fontenc}
\usepackage[latin9]{inputenc}
\usepackage[a4paper]{geometry}
\usepackage{color}
\usepackage{array}
\usepackage{amsmath}
\usepackage{amssymb}
\usepackage{graphicx}
\usepackage[authoryear]{natbib}
\usepackage{ulem}
\usepackage{rotating}
\usepackage[unicode=true,
bookmarks=true,bookmarksnumbered=false,bookmarksopen=true,bookmarksopenlevel=1,
breaklinks=true,pdfborder={0 0 1},backref=section,colorlinks=true]{hyperref}
\usepackage{amsthm}
\usepackage{color}
\usepackage{amsfonts}
\usepackage[title]{appendix}

\setcounter{MaxMatrixCols}{10}
%TCIDATA{OutputFilter=Latex.dll}
%TCIDATA{Version=5.50.0.2953}
%TCIDATA{<META NAME="SaveForMode" CONTENT="1">}
%TCIDATA{BibliographyScheme=Manual}
%TCIDATA{LastRevised=Wednesday, August 06, 2025 11:59:45}
%TCIDATA{<META NAME="GraphicsSave" CONTENT="32">}

\geometry{verbose}
\PassOptionsToPackage{normalem}{ulem}
\hypersetup{
plainpages=false,urlcolor=magenta,citecolor=magenta,linkcolor=blue,pdfstartview=FitH,pdfview=FitH,plainpages=false,urlcolor=blue,citecolor=blue,linkcolor=blue,pdfstartview=FitH,pdfview=FitH}
\makeatletter
\graphicspath{{Figures/}} 
\setlength{\marginparwidth}{0in} \setlength{\marginparsep}{0in}
\setlength{\oddsidemargin}{0in} \setlength{\evensidemargin}{0in}
\setlength{\textwidth}{6.35in} \setlength{\topmargin}{-.50in}
\setlength{\textheight}{9.45in}
\renewcommand{\baselinestretch}{1.2}\small\normalsize
\providecommand{\tabularnewline}{\\}
\newcommand{\ssskip}{\vspace*{0.05cm}}
\newcommand{\sskip}{\vspace*{0.15cm}}
\newcommand{\lskip}{\vspace*{0.45cm}}
\makeatother

%\input{tcilatex}

\begin{document}

\title{\textbf{Wealth Tax and Entrepreneurs: A Simple Model inspired by \cite%
{guvenen2023}}}
\date{\today }
\author{Alessandro Di Nola}
\maketitle

\section{Model}

\begin{itemize}
\item Household model:%
\begin{equation*}
V(a,z)=\max_{c}\left\{ \frac{c^{1-\sigma }}{1-\sigma }+\beta E\left[
V(a^{\prime },z^{\prime })|z\right] \right\} 
\end{equation*}%
subject to%
\begin{equation*}
c+a^{\prime }=Y(a,z)+w(e)\ell ,\ \ \ a^{\prime }\geq 0
\end{equation*}%
\begin{equation*}
Y(a,z)=\left\{ 
\begin{array}{c}
a+\left( \pi \left( a,z\right) +ra\right) \left( 1-\tau _{k}\right) \text{
if CI tax} \\ 
a\left( 1-\tau _{a}\right) +\pi \left( a,z\right) +ra\text{ if wealth tax }%
\end{array}%
\right. 
\end{equation*}%
where CI stands for \textquotedblleft capital income\textquotedblright .

\item Entrepreneurial ability $z$ follows Markov chain with values 
\begin{equation*}
z=[0,z_{L},z_{H}]
\end{equation*}%
and transition matrix $\Pi _{z}$.

\begin{itemize}
\item Households with $z=0$ are normal workers

\item Households with $z=z_{L}$ are \textit{unproductive} entrepreneurs

\item Households with $z=z_{H}$ are \textit{productive} entrepreneurs.
\end{itemize}

\item Entrepreneurial profits $\pi (a,z)$ are given by:%
\begin{equation*}
\pi (a,z)=\max_{k\leq \lambda a}\left\{ zk^{\nu }-(r+\delta )k\right\} 
\end{equation*}%
where $\nu \in (0,1)$ is the span-of-control parameter.
\end{itemize}

\bigskip 

\section{Fiscal Experiments}

\bigskip 

\begin{itemize}
\item Calibrate benchmark model to US\ economy with $\tau _{k}=25\%$.

\item Replace capital income tax with wealth tax $\tau _{a}>0$ is a
revenue-neutral way (in Guvenen paper this requires setting $\tau _{a}=1.2\%$%
).

\item 
\end{itemize}

\end{document}
